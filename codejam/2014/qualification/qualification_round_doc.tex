%% LyX 2.0.7.1 created this file.  For more info, see http://www.lyx.org/.
%% Do not edit unless you really know what you are doing.
\documentclass[american]{article}
\usepackage[T1]{fontenc}
\usepackage[latin9]{inputenc}
\usepackage{geometry}
\geometry{verbose,tmargin=1.5cm,bmargin=1.5cm,lmargin=2cm,rmargin=2cm}
\usepackage{babel}
\usepackage{amstext}
\usepackage{amssymb}
\usepackage[unicode=true,
 bookmarks=true,bookmarksnumbered=false,bookmarksopen=false,
 breaklinks=false,pdfborder={0 0 0},backref=false,colorlinks=false]
 {hyperref}
\hypersetup{
 pdfauthor={Szab� L�r�nt}}

\makeatletter
\@ifundefined{date}{}{\date{}}
%%%%%%%%%%%%%%%%%%%%%%%%%%%%%% User specified LaTeX commands.
\usepackage{indentfirst}

\makeatother

\begin{document}

\title{Google Code Jam 2014}


\author{-- Qualification Round --}

\maketitle

\section{Cookie Clicker}

Translating the problem into mathematical language we have to minimize
the following formula with respect to $k\in\mathbb{N}$:
\[
t(k)=C\sum_{i=0}^{k-1}\frac{1}{2+i\, F}+\frac{X}{2+k\, F}
\]
Note the difference of two consecutive element:
\[
\Delta t(k)=t(k+1)-t(k)=C\left(\sum_{i=0}^{k}\frac{1}{2+i\, F}-\sum_{i=0}^{k-1}\frac{1}{2+i\, F}\right)+X\left(\frac{1}{2+(k+1)\, F}-\frac{1}{2+k\, F}\right)
\]
\[
\Delta t(k)=\frac{C}{2+k\, F}+\frac{X}{2+(k+1)\, F}-\frac{X}{2+k\, F}
\]
Conjecture: $t(k)$ sequence is a strictly monotonic decreasing sequence
for some finite integer $K_{1}\geq0$ and then it's a strictly monotonic
increasing sequence which limit goes to infinity for $k>K_{2}$, so
for at least one $K$ integer $t(k)$ is minimal.

Thus $\Delta t(k<K_{1})<0$ and $\Delta t(k>K_{2})>0$, so we want
to solve $\Delta t(k)=0$
\[
k=\frac{x}{c}-\frac{2}{f}
\]
But this expression does not guarantee that $k$ is integer, so let's
take $k=\lfloor\frac{x}{c}-\frac{2}{f}\rfloor$ %
\footnote{I'm not sure why not rounding, but this do the job (anyway integer
solution should be somewhere around the fraction value)%
}. With a given $k$ we can substitute to the first formula to get
the result.


\section{Deceitful War}

The problem mention that Ken follows an \emph{optimal strategy}. In
this game the optimal strategy is to maximize the won point number.
It is easy to show that trying to ''beat'' every Naomi's weight
with the possible lightest weight and if he can't beat it, letting
it go with his lightest weight can maximize his point -- if they both
play War.


\subsection{War}

According to the previous strategy anything what Naomi call will be
tried to be beaten. Let's investigate what happens if Naomi calls
in ascending order her weights:

When Ken have to use his heaviest weight he won't be able to defeat
more weights, so the remaining number of weights would score for Naomi.
In other order of calls Ken would use the same (if the lightest still
available) or heavier (if he used already the lightest) weights when
beating Naomi's given weight. Reordering Naomi calls doesn't change
(is used the lightest) the final score or decrease it (when not the
lightest was used), but it will no way increase her score.

The steps of a very basic algorithm determining the final score in
War. Basically it's call Naomi's weights in ascending order and check
if Ken can beat it. The ones, which was used in a round become marked.
\begin{enumerate}
\item Sort both of the weight sequences and set unmarked all of the weights
\item $i\leftarrow1$; $score\leftarrow\textrm{length of the sequence}$
\item Select Naomi's $i$th weight
\item $j\leftarrow1$
\item Is Ken's $j$th weight is not marked and heavier than Naomi's $i$th
weight?

\begin{enumerate}
\item Yes: mark Ken's $j$th weight and $score\leftarrow score-1$ and if
$i\leq n$ go back to 3.
\item No: $j\leftarrow j+1$ and if $j\leq n$ go back to 5.
\end{enumerate}
\end{enumerate}

\subsection{Deceitful War}

Naomi knowing Ken's strategy, that he tries to beat every beatable
weight with the possible lightest weight of his, can easily kick out
Ken's heaviest weights with her lighter weight. Let's propose Ken
has $K_{h}$ and Naomi has $N_{l}<K_{h}$ weight. Naomi can lie that
her $N_{l}$ weight is $K_{h}-\Delta$ %
\footnote{Because of the discrete nature of the problem there will be always
infinitely many $\Delta$ for which $K_{h}-\Delta$ only beatable
with $K_{h}$%
}. So Ken would use his $K_{h}$ weight to beat Naomi's $N_{l}$. When
Naomi wants to score with $N_{h}$ she must lie her weight to be the
heaviest in the game (e.g. $\textrm{heaviest}+\Delta$), so Ken would
call his lightest one $K_{l}$. It's optimal for Naomi when she use
his lightest weight to beat Ken's heaviest weight which doesn't exceed
her weight. 

Let's investigate what if Naomi call weights in ascending order. She
must consider in every round that is there any weight which can be
beaten with her currently selected weight. If there's one lie it as
the heaviest weight of the current round$+\Delta$; if there's not
lie it as Ken's heaviest weight$-\Delta$.

A basic algorithm determining the final score in Deceitful War:
\begin{enumerate}
\item Sort both of the weight sequences and set unmarked all of the weights
\item $i\leftarrow1$; $score\leftarrow0$
\item Select Naomi's $i$th weight
\item $j\leftarrow1$
\item Is Ken's $j$th weight is not marked and lighter than Naomi's $i$th
weight?

\begin{enumerate}
\item Yes: mark Ken's $j$th weight and $score\leftarrow score+1$ and if
$i\leq n$ go back to 3.
\item No: $j\leftarrow j+1$ and if $j\leq n$ go back to 5.
\end{enumerate}
\end{enumerate}

\subsection{Generally about War and Deceitful War (without proof)}

Let's denote $U$ Naomi's weight set and $V$ Ken's weight set. Create
$G$ bipartite graph with these $U,V$ sets as vertex sets and connect
with edge $n\in U$ and $k\in V$ according to a game specific rules.
The score will be the maximum bipartite matching of $G$.


\subsubsection{War}

The rule: connect with edge $n\in U$ and $k\in V$ if $n<k$.


\subsubsection{Deceitful War}

The rule: connect with edge $n\in U$ and $k\in V$ if $n>k$.
\end{document}
